\chapter{Introduction}
% \begin{Definition}
%     PoliMi: Politecnico di Milano
% \end{Definition}

%% \smallskip introduces a small vertical skip
%% \bigskip introduces a big vertical skip

%============================= INTRODUCTION =================================
{
    \setlength\epigraphwidth{10cm}
    \setlength\epigraphrule{0pt}
    \renewcommand{\epigraphflush}{center}
    \epigraph{
        I am convinced that machines can and will think. I don't mean that
        machines will behave like men. I don't think for a very long time we
        are going to have a difficult problem distinguishing a man from a
        robot. And I don't think my daughter will ever marry a computer. But I
        think that computer will be doing the things that men do when we say
        they are thinking. I am convinced that machines can and will think in
        our lifetime.}{-- \textup{The Thinking Machine (Artificial Intelligence
            in the 1960s)}, \\ O. Selfridge (Lincoln labs, MIT)}
}

The dream of machines that can think and substitute humans in doing their jobs
dates back to the '60s, if not before. We are still not at that point, although
in the last decade the field experienced outstanding advancements and has been
object of increasing interest. Machine Learning (ML) settled the state of art
in many fields, such as e.g., image classification~\cite{Krizhevsky-2012,
szegedy2016inception, visin2015renet}, semantic
segmentation~\cite{chen2015semantic, visin15}, video
understanding~\citep{srivastava2015unsupervised, Xu-et-al-arxiv2015}, natural
language processing and machine translation~\citep{Bahdanau-et-al-arxiv2014}.
Much of this research is already in commercial products we use every day, such
as e.g., speech recognition and speech synthesis in phones, face detection in
cameras and socials, or traffic signs enhancement in cars. Even more
impressively, a ML algorithm recently won several games of
go~\citep{silver2016alphago} -- a game known for being extremely challenging --
against one of the best human players.

Despite its many successes, machine learning is no lamp genie that can tackle
any problem by simply providing it with enough data. To get results in
machine learning requires a meticulous analysis of the characteristics of the
problem, clever architecture modelling, smart engineering, as well as careful
inspection of complex and extremely nonlinear compositions of transformations.
Most of all, it requires good organization, intuition and patience, since many
of the experiments can last days if not weeks -- even on big clusters of GPUs.

My research is focused on visual scene understanding. My claim is that
understanding a visual scene -- be it an image or a video -- requires to capture
its semantic, and that this has to be done by building an incremental
representation of the context while processing the elements of the observed environment scene. For
this reason I decided to focus on Recurrent Neural Networks (RNNs), a family
of neural networks with memory -- or state -- that can decide autonomously when
to store, retrieve or delete information from their memory.

As a first step to address the problem of image understanding, I focused on
object classification, i.e., the problem of selecting the class an object in a
scene belongs to. Historically, this problem was addressed by hand-engineering
global and local descriptors as characteristic as possible, so that their
presence or absence could be used as a proxy for the presence or absence of a
specific class of objects. From 2012 onwards, handcrafted methods were
abandoned in favour of convolutional neural networks (CNNs), after the
CNN-based model presented in~\cite{Krizhevsky-2012} improved the state of the
art by $10\%$. Since then CNNs-based models dominated the object classification
panorama.

In~\cite{visin2015renet} my co-authors and I presented ReNet, an alternative to
the ubiquitous CNNs for object classification. Our model is based on 4 RNNs
that scan the image in 4 directions. RNNs have the potential to store in their
memory any information that is relevant to retain the context of the part of
the image they have seen up to that moment. The first two RNNs scan each line
of the image reading one pixel (or patch, depending on the configuration) at
the time from left to right and from right to left, respectively. The two
resulting feature maps (i.e., output of each of these RNNs) are concatenated in
each position over the channel axis, yielding a composite feature map where
each position has information on the context of the full row, as in each
position it is a concatenation of an RNN reaching the position from the right
and of an RNN reaching the same position from the left. The second two RNNs
sweep over the composite feature map vertically, top-down and bottom-up
respectively. By reading the composite feature map, each RNN has access in each
position to a "summary" of the corresponding row. Once again, the two feature
maps are concatenated, resulting in a final feature map where each position is
specific to a pixel (or patch) of the image but has information on the full
image. The ReNet architecture allow us to capture the full context of the image
with just one layer (to be fair, two sublayers), as opposed to CNN based
architectures that would need many layers to span the entire image. As usual,
it is still possible to stack multiple ReNet layers to increase the capacity of
the network. ReNet obtained comparable results to the CNN state of the art on
three widely used datasets.

Encouraged by the results of ReNet and the positive feedback from the
scientific community, I worked on a second model based on ReNet, to perform
fine-grained Object Segmentation (i.e., to classify each pixel of the image as
belonging to a specific class). Being able to classify objects without losing
information on their position in the image can be exploited to allow very
precise pixel-level Object Localization, which is essential to many
applications and to a proper understanding of the image.

ReSeg~\citep{visin15} takes advantage of the inner structure of the ReNet
layers that, in contrast to classical convolutional models, allow to propagate
the information through several layers of computation retaining the topological
structure of the input. To speed up training the image is first preprocessed
with a CNN pretrained on big datasets for object classification, to extract
meaningful features and exploit the extra training data. Those rich features
are then processed by several ReNet layers. This results though in an
intermediate feature map that has a smaller resolution than the image. To be
able to classify each pixel, the original resolution has to be recovered. To
this aim, one or more transposed convolutional layers~\citep{dumoulin2016guide}
upsample the feature map to the desired size. This model obtained state of the
art results on three datasets and won the best paper award at the DeepVision
Workshop at CVPR 2016.

The natural next step in the direction of visual scene understanding is the
processing of videos, to exploit the temporal correlation between frames and
improve the performance of the algorithm. It is not trivial to work with videos
in the domain of semantic segmentation: big enough dataset are still lacking
due to the very high cost of labelling each pixel of each frame of a video; in
many cases labels are imprecise and noisy, or missing a well defined semantic
(e.g. "porous" or "vertical mix"), which makes learning harder. Still, it is
a challenging but important problem to tackle and there seems to be room for
improvement w.r.t. the current state of the art. The proposed model combines
the benefits of CNNs -- namely the exploitation of the topological structure in
the images and the processing speed -- and the ability to retain temporal and
context information of RNNs. The paper builds
on~\cite{xingjian2015convolutional} that introduced an RNN whose internal state
is convolutional. This idea is improved by stacking several convolutions inside
the RNN state (as opposed to only one) and by introducing a
\emph{deconvolutional RNN}, whose state is a stack of multiple transposed
convolutions. This model achieved so far state of the art results on two
datasets and encouraging results on a last one.

% As future steps for this research, I plan to conclude the video segmentation
% paper in time to submit it to the CVPR conference 2017. I am also working on
% a follow-up of the ReSeg paper with more experiments and a detailed description
% of the model, to be published as a book chapter in an upcoming book on the
% DeepVision workshop. When this part of the research will be over, we plan
% to focus on two main topics of research: applications of ReSeg-like
% architectures to 3D data to predict depth from images coming from a single
% camera, as well as semantic segmentation in the medical domain. We are already
% in touch with a company working in the healthcare field (Imagia) that should
% provide us with suitable datasets to perform bowel polyps detection and
% classification and potentially commercialize our model.

The rest of this manuscript is organized as follows: \autoref{sec:background}
introduces the most important models and concepts needed to understand the work
done; \autoref{sec:renet} introduces the problem of object classification and
describes in detail the ReNet model and its results; \autoref{sec:reseg}
defines what is referred to as semantic segmentation and how ReSeg tackles that
problem. Finally, \autoref{sec:video_segmentation} moves to video understanding
and specifically video semantic segmentation and highlights the advantages of
convolutional-deconvolutional RNNs in this context. In
\autoref{sec:conclusion} summarizes the main contribution of this research and
proposes some of the many possible future directions of research that can build
on top of this work.
